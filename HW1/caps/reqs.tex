\section{Database Requirements} \label{cap:reqs}

\subsection{Assumptions}
    \begin{enumerate}
        \item Purchase order is placed for one Ingredient at a time with vendor (that is bill for purchase of each Ingredient from vendor (maybe same) are separate).
        \item All the attributes are NOT NULL unless specified specifically.
        \item Assuming that each customer places just one order. If the same customer places another order than it is treated as a separate customer.
        \item Each customer can take up/reserve only one table and not multiple tables.
    \end{enumerate}

\subsection{Strong Entity Types}
Note: Attributes in red color serves as primary key for that entity type. \\
\textcolor{red}{Red Colored attributes : Primary Key} \\
\textcolor{darkgreen}{Green Colored attributes : Foreign Key}
    \begin{enumerate}
        \item Employee \\
        \textit{Stores the \underline{details of all the employees} (cooks, waiters, cleaning staff and managers) working in the restaurant.}
            \begin{enumerate}[label=\alph*.]
                \item\textcolor{red}{Employee Id}
                    \begin{itemize}[label=-]
                        \item \textcolor{gray}{Data Type : Integer}
                    \end{itemize}
                \item Name (Composite Attribute : First Name + Middle Name + Last Name)
                    \begin{itemize}[label=-]
                        \item \textcolor{gray}{Data Type : Varchar(String)}
                    \end{itemize}
                \item Age (Derivable)
                    \begin{itemize}[label=-]
                        \item \textcolor{gray}{Data Type : Integer}
                        \item \textcolor{gray}{Must be >= 18}
                        \item \textcolor{gray}{Age = Curr Date - DOB}
                    \end{itemize}
                \item DOB
                    \begin{itemize}[label=-]
                        \item \textcolor{gray}{Data Type : Date}
                    \end{itemize}
                \item Gender
                    \begin{itemize}[label=-]
                        \item \textcolor{gray}{Data Type : ENUM}
                    \end{itemize}
                \item Address (Complex) (Composite Attribute : Apt no + Street no + area + city + state)
                    \begin{itemize}[label=-]
                        \item \textcolor{gray}{Data Type : Varchar(String)}
                    \end{itemize}
                \item Contact No
                    \begin{itemize}[label=-]
                        \item \textcolor{gray}{Data Type : Integer}
                        \item \textcolor{gray}{10 digits}
                    \end{itemize}
                \item Role
                    \begin{itemize}[label=-]
                        \item \textcolor{gray}{Data Type : ENUM}
                    \end{itemize}
                \item Start Date
                    \begin{itemize}[label=-]
                        \item \textcolor{gray}{Data Type : Date}
                    \end{itemize}
                \item End Date
                    \begin{itemize}[label=-]
                        \item \textcolor{gray}{Data Type : Date}
                        \item \textcolor{gray}{NULLABLE}
                    \end{itemize}
                \item Salary
                    \begin{itemize}[label=-]
                        \item \textcolor{gray}{Data Type : Integer}
                        \item \textcolor{gray}{Currency : Beli}
                    \end{itemize}
            \end{enumerate}
        
        \item Customer \\
        \textit{Stores the \underline{details of all the customers} that visit the restaurant.}
            \begin{enumerate}[label=\alph*.]
                \item\textcolor{red}{Customer Id}
                    \begin{itemize}[label=-]
                        \item \textcolor{gray}{Data Type : Integer}
                    \end{itemize}
                \item Name (Composite Attribute : First Name + Middle Name + Last Name)
                    \begin{itemize}[label=-]
                        \item \textcolor{gray}{Data Type : Varchar(String)}
                    \end{itemize}
                \item Contact No
                    \begin{itemize}[label=-]
                        \item \textcolor{gray}{Data Type : Integer}
                        \item \textcolor{gray}{10 digits}
                    \end{itemize}
                \item Favourite Dishes (Multivalued)
                    \begin{itemize}[label=-]
                        \item \textcolor{gray}{Data Type : Varchar(String of comma separated values)}
                    \end{itemize}
            \end{enumerate}

        \item Vendor \\
        \textit{Stores the \underline{details of all the vendors} from which the restaurant purchases the ingredients.}
            \begin{enumerate}[label=\alph*.]
                \item\textcolor{red}{Vendor Id}
                    \begin{itemize}[label=-]
                        \item \textcolor{gray}{Data Type : Integer}
                    \end{itemize}
                \item Name (Composite Attribute : First Name + Middle Name + Last Name)
                    \begin{itemize}[label=-]
                        \item \textcolor{gray}{Data Type : Varchar(String)}
                    \end{itemize}
                \item Contact No
                    \begin{itemize}[label=-]
                        \item \textcolor{gray}{Data Type : Integer}
                        \item \textcolor{gray}{10 digits}
                    \end{itemize}
                \item Address (Complex) (Composite Attribute : Apt no + Street no + area + city + state)
                    \begin{itemize}[label=-]
                        \item \textcolor{gray}{Data Type : Varchar(String)}
                    \end{itemize}
            \end{enumerate}
        
        \item Order \\
        \textit{Stores information about \underline{individual orders.}}
            \begin{enumerate}[label=\alph*.]
                \item\textcolor{red}{Order Id}
                    \begin{itemize}[label=-]
                        \item \textcolor{gray}{Data Type : Integer}
                    \end{itemize}
                \item Date and time
                    \begin{itemize}[label=-]
                        \item \textcolor{gray}{Data Type : Date-Time}
                    \end{itemize}
                \item \textcolor{darkgreen}{Customer Id}
                    \begin{itemize}[label=-]
                        \item \textcolor{gray}{Data Type : Integer}
                    \end{itemize}
                \item Dishes (Multivalued)
                    \begin{itemize}[label=-]
                        \item \textcolor{gray}{Data Type : Varchar(String of comma separated valued)}
                    \end{itemize}
                \item Table No
                    \begin{itemize}[label=-]
                        \item \textcolor{gray}{Data Type : Integer}
                    \end{itemize}
            \end{enumerate}
        
        \item Dishes \\
        \textit{Stores information about the \underline{dishes being served}, ingredients required in the dishes, cuisine of the dishes and the cook which prepares the dish.}
            \begin{enumerate}[label=\alph*.]
                \item\textcolor{red}{Dish Id}
                    \begin{itemize}[label=-]
                        \item \textcolor{gray}{Data Type : Integer}
                    \end{itemize}
                \item Name
                    \begin{itemize}[label=-]
                        \item \textcolor{gray}{Data Type : Varchar(String)}
                    \end{itemize}
                \item \textcolor{darkgreen}{Ingredient Id}
                    \begin{itemize}[label=-]
                        \item \textcolor{gray}{Data Type : Integer}
                    \end{itemize}
                \item \textcolor{darkgreen}{Employee Id (Id of cook)}
                    \begin{itemize}[label=-]
                        \item \textcolor{gray}{Data Type : Integer}
                    \end{itemize}
                \item Cuisine
                    \begin{itemize}[label=-]
                        \item \textcolor{gray}{Data Type : ENUM}
                    \end{itemize}
                \item Price
                    \begin{itemize}[label=-]
                        \item \textcolor{gray}{Data Type : Decimal}
                        \item \textcolor{gray}{Upto 2 decimal places}
                        \item \textcolor{gray}{Currency : Beli}
                    \end{itemize}
                \item Discount
                    \begin{itemize}[label=-]
                        \item \textcolor{gray}{Data Type : Decimal}
                        \item \textcolor{gray}{Upto 2 decimal places}
                        \item \textcolor{gray}{Unit : \%}
                    \end{itemize}
            \end{enumerate}

        \item Inventory \\
        \textit{Stores information about \underline{availability of ingredients} and in what quantities.}
            \begin{enumerate}[label=\alph*.]
                \item\textcolor{red}{Ingredient Id}
                    \begin{itemize}[label=-]
                        \item \textcolor{gray}{Data Type : Integer}
                    \end{itemize}
                \item Name
                    \begin{itemize}[label=-]
                        \item \textcolor{gray}{Data Type : Varchar(String)}
                    \end{itemize}
                \item \textcolor{darkgreen}{Vendor Id}
                    \begin{itemize}[label=-]
                        \item \textcolor{gray}{Data Type : Integer}
                    \end{itemize}
                \item Quantity
                    \begin{itemize}[label=-]
                        \item \textcolor{gray}{Data Type : Integer}
                        \item \textcolor{gray}{Unit : Kg/pcs (based on item)}
                    \end{itemize}
                \item Price
                    \begin{itemize}[label=-]
                        \item \textcolor{gray}{Data Type : Decimal}
                        \item \textcolor{gray}{Upto 2 places precision}
                        \item \textcolor{gray}{Currency : Beli}
                    \end{itemize}
            \end{enumerate}
        
        \item Bills \\
        \textit{Stores information about \underline{bills of all orders} placed by customers.}
            \begin{enumerate}[label=\alph*.]
                \item\textcolor{red}{Bill Id}
                    \begin{itemize}[label=-]
                        \item \textcolor{gray}{Data Type : Integer}
                    \end{itemize}
                \item \textcolor{darkgreen}{Order Id}
                    \begin{itemize}[label=-]
                        \item \textcolor{gray}{Data Type : Integer}
                    \end{itemize}
                \item \textcolor{darkgreen}{Customer Id}
                    \begin{itemize}[label=-]
                        \item \textcolor{gray}{Data Type : Integer}
                    \end{itemize}
                \item Total Amount
                    \begin{itemize}[label=-]
                        \item \textcolor{gray}{Data Type : Decimal}
                        \item \textcolor{gray}{Upto 2 places of precision}
                    \end{itemize}
                \item Date and time
                    \begin{itemize}[label=-]
                        \item \textcolor{gray}{Data Type : Date-Time}
                    \end{itemize}
                \item \textcolor{darkgreen}{Table No}
                    \begin{itemize}[label=-]
                        \item \textcolor{gray}{Data Type : Integer}
                    \end{itemize}
                \item Mode Of Payment
                    \begin{itemize}[label=-]
                        \item \textcolor{gray}{Data Type : ENUM}
                    \end{itemize}
            \end{enumerate}
        
        \item Tables \\
        \textit{Stores Information about \underline{tables}.}
            \begin{enumerate}[label=\alph*.]
                \item\textcolor{red}{Table No}
                    \begin{itemize}[label=-]
                        \item \textcolor{gray}{Data Type : Integer}
                    \end{itemize}
                \item Capacity
                    \begin{itemize}[label=-]
                        \item \textcolor{gray}{Data Type : Integer}
                    \end{itemize}
                \item Used
                    \begin{itemize}[label=-]
                        \item \textcolor{gray}{Data Type : ENUM}
                    \end{itemize}
                \item Reserved
                    \begin{itemize}[label=-]
                        \item \textcolor{gray}{Data Type : ENUM}
                    \end{itemize}
            \end{enumerate}
        
        \item Restock Orders \\
        \textit{Stores information about the \underline{orders placed with vendors} for the purchase of ingredients.}
            \begin{enumerate}[label=\alph*.]
                \item\textcolor{red}{Restock Order Id}
                    \begin{itemize}[label=-]
                        \item \textcolor{gray}{Data Type : Integer}
                    \end{itemize}
                \item \textcolor{darkgreen}{Ingredient Id}
                    \begin{itemize}[label=-]
                        \item \textcolor{gray}{Data Type : Integer}
                    \end{itemize}
                \item \textcolor{darkgreen}{Vendor Id}
                    \begin{itemize}[label=-]
                        \item \textcolor{gray}{Data Type : Integer}
                    \end{itemize}
                \item Quantity
                    \begin{itemize}[label=-]
                        \item \textcolor{gray}{Data Type : Integer}
                    \end{itemize}
                \item Total Cost
                    \begin{itemize}[label=-]
                        \item \textcolor{gray}{Data Type : Decimal}
                        \item \textcolor{gray}{With 2 digits of precison}
                        \item \textcolor{gray}{Currency : Beli}
                    \end{itemize}
                \item Paid
                    \begin{itemize}[label=-]
                        \item \textcolor{gray}{Data Type : ENUM}
                    \end{itemize}
            \end{enumerate}
    \end{enumerate}

\subsection{Weak Entity Types}
    \begin{enumerate}
        \item Feedback \\
        \textit{Stores Information about the \underline{feedbacks given by customers} for their orders.}
            \begin{enumerate}[label=\alph*.]
                \item \textcolor{darkgreen}{Customer Id}
                    \begin{itemize}[label=-]
                        \item \textcolor{gray}{Data Type : Integer}
                    \end{itemize}
                \item Rating
                    \begin{itemize}[label=-]
                        \item \textcolor{gray}{Data Type : ENUM}
                    \end{itemize}
                \item Date
                    \begin{itemize}[label=-]
                        \item \textcolor{gray}{Data Type : Date}
                    \end{itemize}
                \item \textcolor{darkgreen}{Order Id}
                    \begin{itemize}[label=-]
                        \item \textcolor{gray}{Data Type : Integer}
                    \end{itemize}
                \item Review
                    \begin{itemize}[label=-]
                        \item \textcolor{gray}{Data Type : Varchar(String)}
                    \end{itemize}
            \end{enumerate}
        
        \item Reservations \\
        \textit{Stores information about the \underline{tables reserved by customers}.}
            \begin{enumerate}[label=\alph*.]
                \item \textcolor{darkgreen}{Customer Id}
                    \begin{itemize}[label=-]
                        \item \textcolor{gray}{Data Type : Integer}
                    \end{itemize}
                \item \textcolor{darkgreen}{Table No}
                    \begin{itemize}[label=-]
                        \item \textcolor{gray}{Data Type : Integer}
                    \end{itemize}
                \item From
                    \begin{itemize}[label=-]
                        \item \textcolor{gray}{Data Type : Date-Time}
                    \end{itemize}
                \item Till
                    \begin{itemize}[label=-]
                        \item \textcolor{gray}{Data Type : Date-Time}
                        \item \textcolor{gray}{NULLABLE}
                    \end{itemize}
            \end{enumerate}
    \end{enumerate}

\subsection{Relationship Types}
    \subsubsection{Binary Relationships}
        \begin{itemize}
            \item Waiter \textbf{SERVES} the Order
                \begin{itemize}[label=-]
                    \item Participating entities : Waiter (Employee), Order
                    \item Cardinality Ratio : 1:N
                \end{itemize}

            \item Chef \textbf{PREPARES} the Order
                \begin{itemize}[label=-]
                    \item Participating entities : Chef (Employee), Order
                    \item Cardinality Ratio : N:M
                \end{itemize}

            \item Vendor \textbf{SELLS} Ingredients
                \begin{itemize}[label=-]
                    \item Participating entities : Vendor, Ingredient
                    \item Cardinality Ratio : 1:N
                \end{itemize}

            % \item Customer \textbf{PAYS} Bill
            %     \begin{itemize}[label=-]
            %         \item Participating entities : Customer, Bill
            %         \item Cardinality Ratio : 1:1
            %     \end{itemize}

            \item Manager \textbf{MANAGES} Inventory
                \begin{itemize}[label=-]
                    \item Participating entities : Manager, Inventory
                    \item Cardinality Ratio : 1:1
                \end{itemize}

            \item Manager \textbf{OVERSEES} Employees
                \begin{itemize}[label=-]
                    \item Participating entities : Manager, Employee
                    \item Cardinality Ratio : 1:N
                \end{itemize}

            % \item Customer \textbf{RATES} Order
            %     \begin{itemize}[label=-]
            %         \item Participating entities : Customer, Order
            %         \item Cardinality Ratio : 1:1
            %     \end{itemize}

            \item Cook \textbf{REPORTS} to Manager
                \begin{itemize}[label=-]
                    \item Participating entities : Cook, Manager
                    \item Cardinality Ratio : N:1
                \end{itemize}

            \item Waiter \textbf{REPORTS} to Manager
                \begin{itemize}[label=-]
                    \item Participating entities : Waiter, Manager
                    \item Cardinality Ratio : N:1
                \end{itemize}

            \item Cleaning Staff \textbf{REPORTS} to Manager
                \begin{itemize}[label=-]
                    \item Participating entities : Cleaning Staff, Manager
                    \item Cardinality Ratio : N:1
                \end{itemize}

            % \item Manager \textbf{PLACES} Restock Order
            %     \begin{itemize}[label=-]
            %         \item Participating entities : Manager, Restock Order
            %         \item Cardinality Ratio : 1:N
            %     \end{itemize}

                \item Customer \textbf{RESERVES} Table
                \begin{itemize}[label=-]
                    \item Participating entities : Customer, Table
                    \item Cardinality Ratio : 1:1
                \end{itemize}

            \item Customer \textbf{SITS} on a Table
                \begin{itemize}[label=-]
                    \item Participating entities : Customer, Table
                    \item Cardinality Ratio : 1:1
                \end{itemize}

            % \item Customer \textbf{ORDERS} an Order
            %     \begin{itemize}[label=-]
            %         \item Participating entities : Customer, Order
            %         \item Cardinality Ratio : 1:1
            %     \end{itemize}

            \item Dish \textbf{REQUIRES} Ingredient
                \begin{itemize}[label=-]
                    \item Participating entities : Dish, Ingredient
                    \item Cardinality Ratio : N:M
                \end{itemize}

            \item Customer \textbf{RATES} Order
                \begin{itemize}[label=-]
                    \item Participating entities : Customer, Order
                    \item Cardinality Ratio : 1:1
                \end{itemize}
        \end{itemize}
    \subsubsection{Degree $>$ 2 relationship types}
        \begin{itemize}
            \item Customer \textbf{PAYS} Bill for an Order and \textbf{GIVES} Feedback
                \begin{itemize}[label=-]
                    \item Participating entities : Customer, Bill, Order, Feedback
                    \item Cardinality Ratio : 1:1:1:1
                \end{itemize}

            \item Manager \textbf{PLACES} Restock Order with Vendor
                \begin{itemize}[label=-]
                    \item Participating entities : Manager, Restock Order, Vendor
                    \item Cardinality Ratio : 1:M:N
                \end{itemize}
        \end{itemize}